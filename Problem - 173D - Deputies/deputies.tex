\documentclass[12pt, a4paper]{article}
\usepackage{algpseudocode,algorithm,algorithmicx}
\usepackage[usenames,dvipsnames,svgnames,table]{xcolor}
\usepackage{hyperref}
\usepackage{graphicx}

\begin{document}
    \title{\textcolor{blue}{\huge\textbf{Problem - 173D - Codeforces - Deputies}}\\}
	\author{\large{Javier E. Domínguez Hernández C-412}}  
    \date{7 de julio de 2022}
    \maketitle

	\section{Descripción}

	\subsection{English}
		The Trinitarian kingdom has exactly n = 3k cities. All of them are located on the shores of river Trissisipi, which flows through the whole kingdom. Some of the cities are located on one side of the river, and all the rest are on the other side.\\

		Some cities are connected by bridges built between them. Each bridge connects two cities that are located on the opposite sides of the river. Between any two cities exists no more than one bridge.\\

		The recently inaugurated King Tristan the Third is busy distributing his deputies among cities. In total there are k deputies and the king wants to commission each of them to control exactly three cities. However, no deputy can be entrusted to manage the cities, which are connected by a bridge — the deputy can set a too high fee for travelling over the bridge to benefit his pocket, which is bad for the reputation of the king.\\

		Help King Tristan the Third distribute the deputies between the cities, if it is possible.\\


		{\bf Input}\\

		The first line contains two integers n and m — the number of cities and bridges ($3 \le n < 105, n = 3k, 0 \le m \le 105$). Next $m$ lines describe the bridges. The $i-th$ line contains two integers $a_i$ and $b_i$ — the numbers of cities that are connected by the $i-th$ bridge (1 $ \le a_i , bi \le n, a_i \neq b_i, 1 \le  i \le m$).\\

		It is guaranteed that no bridge connects a city with itself and that any two cities are connected with no more than one bridge.\\

		{\bf Output}\\ 

		If distributing the deputies in the required manner is impossible, print in a single line "NO" (without the quotes).\\

		Otherwise, in the first line print "YES" (without the quotes), and in the second line print which deputy should be put in charge of each city. The $i-th$ number should represent the number of the deputy (from $1$ to $k$), who should be in charge of city numbered $i-th$ in the input — overall there should be $n$ numbers.\\

		If there are multiple solutions, print any of them.\\
		\newpage

	\subsection{Español}
		El reino trinitario tiene exactamente $n = 3k$ ciudades. Todos ellos están situados a orillas del río Trissisipi, que atraviesa todo el reino. Algunas de las ciudades están ubicadas a un lado del río, y todas las demás están al otro lado.\\

		Algunas ciudades están conectadas por puentes construidos entre ellas. Cada puente conecta dos ciudades que se encuentran en lados opuestos del río. Entre dos ciudades cualesquiera no existe más de un puente.\\

		El recién investido rey Tristán III está ocupado repartiendo sus diputados entre las ciudades. En total hay $k$ diputados y el rey quiere encargar a cada uno de ellos que controle exactamente tres ciudades. Sin embargo, no se puede confiar a ningún diputado la gestión de ciudades que estén conectadas por un puente; el diputado puede establecer una tarifa demasiado alta para viajar por el puente en beneficio de su bolsillo, lo que es malo para la reputación del rey.\\

		Ayuda al rey Tristán III a distribuir los diputados entre las ciudades, si es posible.\\

		{\bf Entrada}\\

		La primera línea contiene dos números enteros $n$ y $m$: el número de ciudades y puentes ($3 \le n < 105, n = 3k, 0 \le m \le 105$). Las siguientes $m$ líneas describen los puentes. La $i-ésima$ línea contiene dos números enteros $a_i$ y $b_i$: el número de ciudades que están conectadas por el $i-ésimo$ puente ($1 \le a_i, b_i \le n, a_i \neq b_i, 1 \le i \le m$).\\

		Se garantiza que ningún puente conecta una ciudad consigo misma y que dos ciudades cualesquiera están conectadas con no más de un puente.\\

		{\bf Salida}\\

		Si no es posible distribuir los diputados en la forma requerida, escriba en una sola línea "NO" (sin las comillas).\\

		De lo contrario, en la primera línea escriba "SÍ" (sin las comillas), y en la segunda línea escriba qué diputado debe estar a cargo de cada ciudad. El número $i-ésimo$ debe representar el número del diputado (del $1$ al $k$), que debe estar a cargo de la $i-ésima$ ciudad dada en la entrada; en general, debe haber $n$ números.\\

		Si hay varias soluciones, imprima cualquiera de ellas.
		\newpage

\end{document}
