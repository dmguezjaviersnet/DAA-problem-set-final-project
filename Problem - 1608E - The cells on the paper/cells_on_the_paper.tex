\documentclass[12pt, a4paper]{article}
\usepackage{algpseudocode,algorithm,algorithmicx}
\usepackage[usenames,dvipsnames,svgnames,table]{xcolor}
\usepackage{hyperref}
\usepackage{graphicx}

\begin{document}
    \title{\textcolor{blue}{\huge\textbf{Problem - 173D - Codeforces - Deputies}}\\}
	\author{\large{Javier E. Domínguez Hernández C-412}}  
    \date{7 de julio de 2022}
    \maketitle

	\section{Descripción}

	\subsection{English}
		On an endless checkered sheet of paper, $n$ cells are chosen and colored in three colors, where $n$ is divisible by $3$. It turns out that there are exactly $\frac{n}{3}$ marked cells of each of three colors!.\\

		Find the largest $k$ such that it's possible to choose $\frac{k}{3}$ cells of each color, remove all other marked cells, and then select three rectangles with sides parallel to the grid lines so that the following conditions hold:\\

		\begin{enumerate}
			\item No two rectangles can intersect (but they can share a part of the boundary). In other words, the area of intersection of any two of these rectangles must be $0$.\\

			\item The $i-th$ rectangle contains all the chosen cells of the $i-th$ color and no chosen cells of other colors, for $i=1, 2, 3$.\\
		\end{enumerate}		

		{\bf Input}\\

		The first line of the input contains a single integer $n$ — the number of the marked cells ($3 \le n \le 105$, $n$ is divisible by $3$).\\

		The $i-th$ of the following $n$ lines contains three integers $xi$, $yi$, $ci$ ($|xi|, |yi| \le 109; 1 \le c_i \le 3$), where ($xi$, $yi$) are the coordinates of the $i-th$ marked cell and $c_i$ is its color.\\

		It's guaranteed that all cells ($xi$, $yi$) in the input are distinct, and that there are exactly $\frac{n}{3}$ cells of each color.\\

		{\bf Output}\\

		Output a single integer $k$ — the largest number of cells you can leave.\\
		\newpage

	\subsection{Español}
	En una hoja de papel a cuadros sin fin, se eligen $n$ celdas y se colorean en tres colores, donde $n$ es divisible por $3$. ¡Resulta que hay exactamente $\frac{n}{3}$ celdas marcadas de cada uno de los tres colores!.\\

	Encuentre el mayor $k$ tal que sea posible elegir $\frac{k}{3}$ celdas de cada color, eliminar todas las demás celdas marcadas y luego seleccionar tres rectángulos con lados paralelos a las líneas de la cuadrícula y que se cumplan las siguientes condiciones:\\

	\begin{enumerate}
		\item Dos rectángulos no pueden cruzarse (pero pueden compartir una parte del límite). En otras palabras, el área de intersección de cualquiera de estos dos rectángulos debe ser $0$.\\

		\item El $i-ésimo$ rectángulo contiene todas las celdas elegidas del $i-ésimo$ color y ninguna celda elegida de otros colores, para $i=1,2,3$.\\
	\end{enumerate}
	
		{\bf Entrada}\\
		La primera línea de la entrada contiene un único número entero $n$: el número de celdas marcadas ($3 \le n \le 105$, $n$ es divisible por $3$).\\

		La $i-ésima$ de las siguientes $n$ líneas contiene tres enteros $x_i$, $y_i$, $c_i$ ($|x_i|, |y_i| \le 109; 1 \le c_i \le 3$), donde ($x_i$, $y_i$) son las coordenadas de la $i-ésima$ marcada celda y $c_i$ es su color.\\

		Se garantiza que todas las celdas ($xi$, $yi$) en la entrada sean distintas y que haya exactamente $\frac{n}{3}$ celdas de cada color.\\

		{\bf Salida}\\
		Muestra un solo entero $k$: el mayor número de celdas que puedes dejar.\\

\end{document}
